\section{Билет 5}
\section*{Алгоритм Рабина-Карпа}
\begin{enumerate}
    \item Подсчитать $h(s[0 \ldots m-1])$ и $h(w[0 \ldots m-1])$, а также $p^m$, для ускорения ответов на запрос.
    \item Для $i \in [0 \ldots n-m]$ вычислить $h(s[i \ldots i+m-1])$ и сравнить с $h(w[0 \ldots m-1])$. Если используется полиномиальный хэш, то $h(s[i \ldots i+m-1]) = (h(s[i-1 \ldots i+m-2]) - s[i-1]) \cdot p + s[i+m-1]$. Если хэши равны, то образец $w$ скорее всего содержится в строке $s$ начиная с позиции $i$, хотя возможны и ложные срабатывания алгоритма.
\end{enumerate}
\textbf{Асимптотика}
\newline
Изначальный подсчёт хешей выполняется за $O(m)$. Каждая итерация выполняется за $O(1)$, В цикле всего $n-m+1$ итераций. Итоговое время работы алгоритма $O(n+m)$. Однако, если требуется исключить ложные срабатывания алгоритма полностью, то придется проверить все полученные позиции вхождения на истинность и тогда в худшем случае итоговое время работы алгоритма будет $O(n \cdot m$).