\section{Билет 4}
\section*{Полиномиальный хеш}
\textbf{Полиномиальный хэш} $h$~---~число $h=hash(s[0..n-1])= (p^{n-1} \cdot s[0]+ \ldots +p^0 \cdot s[n-1]) \% M$, где $p, M$~---~некоторые простые числа, а $s[i]$~---~код $i$-ого символа строки $s$. \newline
\textbf{Проверка подстрок данной строки $s$ на равенство с помощью хэшей за $O(1)$ на запрос и $O(|s|)$ предподсчёта.}
\newline
Сделаем предподсчёт степеней $p$ и $h(s[0:i])$. Предподсчёт можно делать по схеме Горнера $h(s[0:n]) = ((((a_0 \cdot p + a_1) \cdot p + a_2) \cdot p ... ) \cdot p + a_{n - 1}$. Тогда $h(s[l:r]) = h(s[o:r)) - h(s[0:l)) \cdot p^{r-l+1}$
\newline
Тогда при запросе на сравнение подстрок можно сравнивать не подстроки, а их хэши.
\newline
Предпосчёт занял $O(|s|)$, ответ на запрос будет $O(1)$.