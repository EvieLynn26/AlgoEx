\section{Билет 2.}
\section*{префикс функция: определение и линейный алгоритм вычисления}
\par
\textbf{Префикс функция} от строки $S$ и позиции $x$~--- это длина наидлиннейшей
подтроки не являющейся префиксом, заканчивающейся в позиции $x$ и совпадающей с префиксом строки $S$.
Далее $P_S(x)$. Из определения следует, что если $S.lenght() \geq 1$, то $P_S(0)$
определен и равен \textbf{0}.

\par
Вычисление префикс функции~--- задача о нахождении всех значений, иначе говоря,
найти $P_S(x)$ для всех $x \in \overline{1, S.lenght() - 1}$.

\subsection*{Алгоритм вычисления префикс функции}

\par
Пусть мы нашли значения префикс функции для $\forall x \in \overline{1, i - 1}$.
Отметим, что $P_S(i) \leq P_S(i - 1) + 1$, причем можно утвержать, что
$P_S(i) = P_S(i-1) + 1$ лишь тогда, когда $S[i] = S[P_S(i-1)]$. Пусть это условие
не выполнено. Рассмотрим $y_1 = P_S(i - 1) - 1$. Из определения префикс функции
следует, что префикс, заканчивающийся в позиции $y_1$ совпадает с наидлиннейшей
подстрокой заканчивающейся в позиции $i - 1$. Поэтому, если $S[i] = S[P_S(y_1)]$,
то исходя из предыдущих рассуждений, $P_S(i) = P_S(y_1) + 1$.
В случае, если $S[i] \neq S[P_S(y_1)]$, будем рассматривать $y_n = P_S(y_{n-1}) - 1$
такое, что $y_n \geq 0, S[i] = S[P_S(y_n)]$ и $n$~--- минимально.
Так как $P_S(y_n)$~--- убывающая последовательность, при найденном $y_n$
можно заключить, что $P_S(i) = P_S(y_n) + 1$. В случае, если такого $y_n$ не
нашлось, $P_S(i) = 0$.

\subsection*{Оценим время работы}

Префикс функция на каждом шаге либо возрастает на единицу, либо невозрастает,
линейность алгоритма очевидна.