\section{Билет 3.}
\section*{Алгоритм КМП, решение за O(pattern.length()) памяти и O(pattern.length() + text.length()) времени}
\par
Алгоритм \textbf{Кнута-Мориса-Пратта}~--- алгоритм, позволяющий
искать в строке $S$ паттерн $T$ за линейное время.

Объединим алфавиты строк $S$ и $T$ в алфавит $\Omega$. Пусть \$~--- 
какой-то символ, которого нет в объединении. Пусть $\overline{T} = T + \$$.
Вычислим префикс функцию для $\overline{T}$, это будет
массив $P$.
\par
Рассмотрим $S^* = \overline{T} + S'$, где $S'$~--- какая-то строка
с символами из $\Omega$. Тогда $\forall x \in \overline{0, S^*.length() - 1}$ выполняется $ 
P_S(x) \leq T.length()$.
\par
Таким образом, чтобы вычилсять префикс функцию для суффикса $S^*$ с позиции
$\overline{T}.length()$ длины $O(1)$ будет достаточно $O(1)$ памяти. Если
хранить значения префикс функции лишь в момент рассмотрения, то для суффикса
любой длины потребуется $O(1)$ памяти.
\par
Рассмотрим $M = \overline{T} + S$. Паттерн входит в $S$ и заканчивается
в некоторой позиции $x$ тогда и только тогда, когда $P_M(x + \overline{T}.length()) = T.length()$.
Этот критерий следует из определения префикс функции и из построения $\overline{T}$.
Пользуясь этим критерием, можно находить позиции вхождения паттерна в $S$.

\subsection*{Оценим время работы}

Построение $P$ потребует $O(T.length())$ времени и $O(T.length())$ памяти.
Нахождение вхождений паттерна потребует $O(S.length())$ времени и $O(1)$ памяти.
Итоговое время работы $O(T.lenght() + S.lenght())$, а требуемая память
$O(T.lenght())$.